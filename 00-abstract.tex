Thanks to the advance of device scaling technologies, an ultra-scale SSD 
(\eg{} 32TB) is introduced. This rapid increase of SSD capacity comes
at the cost of a huge index table (\eg{} 32GB) 
to map logical blocks to physical sectors.
%Instead of keeping such a huge table in DRAM,
Many have proposed various caching strategies 
to provide reasonable performance using small DRAM.
But, they provide seriously deteriorated read latency if
workloads have weak locality or when storage space is fragmented.  
This paper proposes a novel approximate LSM-tree index structure,
\textit{\ours{}}. 
\ours{} uses an LSM-tree as an index structure
and creates approximate indices over the tree.
Instead of relying on caching, it loads tiny approximate
indices (3--7-bit per index) entirely in DRAM and uses them to locate data.
%By balancing a tree hierarchy, it also offers sufficiently high write throughput.
\ours{} is never affected by locality and system's condition.
Even under random read workloads running on severely fragmented storage space,
\ours{} exhibits read latency close to when the entire index table is
kept in large DRAM. This performance is achieved by using only 29.1\% of DRAM
that the index table needs.
