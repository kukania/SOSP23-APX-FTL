
%\vspace{-10pt}
\section{Conclusion}
\label{sec:con}

This paper presented the approximate indexing technique
based on FP and PLR for ultra-scale SSDs.  
By utilizing the memory-efficient property 
of these two algorithms, we designed \ours{}
that guaranteed RAF under 1.1 while only using
29.1\% of DRAM over the optimal FTL that loads its index table
entirely in DRAM.
Our results showed that \ours{} outperformed the demand-based indexing
on read latency as well as throughput. In realistic benchmarks,
\ours{} showed short read latency 
close to that of the optimal FTL.
The overall throughput of \ours{} was 32\% higher than 
the demand-based indexing, on average.
%Overall, \ours{} offered \JS{45\%} shorter latency and 
%\JS{20\%} higher throughput than demand-based FTLs, on average.
As future work, we plan to optimize the memory usage of the shortcut table 
that accounts for the majority of DRAM usage.

\begin{comment}
In write performance, \ours{} performs 20\% lower WAS in average than other 
demand-based FTLs on realistic benchmarks. But, it shows lower 
write throughput (up to 21\% lower) on random write workloads.
\end{comment}

\begin{comment}
In this paper, we proposed a new probability-based address translation
algorithm, \ours{}, which was based on bloom filters.  Compared to the
sector-level FTL that maintains all the mapping entries in DRAM, \ours{}
required only 21\% of DRAM, but guaranteed a low enough RAF which was less than
1.1. 
\ours{} offered a slightly higher WAF than other FTLs, but it was
regulated under 3.0.  As future work, we plan to optimize the Reblooming
process to minimize its overhead, implement HW accelerator for membership
tests, and carry out comprehensive experiments using real-world workloads.
\end{comment}
%\vspace{-10pt}
%\section*{Acknowledgements}
%HiSilicon Technologies Co., Ltd. supported this research.  This study was also
%supported by the NRF grant funded by the Korea government (Ministry of Science
%and ICT) (NRF-2018R1A5A1060031). 
\begin{comment}
    

{
\renewcommand{\arraystretch}{0.4}
\begin{table*}[t]
    \small
    \caption{\fixme{RAF, Lookup and WAF cost of algorithms}}
    \vspace{-10pt}
    \centering
    \begin{tabular}{|c||c|c|c|}
        \hline
        {FTL} &  {RAF}  & {Lookup}  & {WAF (simulation?)}  \\\hline
        {Hybrid} & {1}  & {$O(n_{log})$} & {$(1-a_{hit,log})\cdot SPB \cdot \frac{1}{blocknum} \cdot SPB$} \\\hline
        {$\mu$FTL} & {$1+\sum_{i=1}^{h}\cdot(1-\alpha_{hit,i})$} & {$O(h\cdot log_{2}(n_{entry}))$} & {$1+\alpha_{dirty-evict}\cdot SPP$} \\\hline
        {DFTL} & {$1+(1-\alpha_{hit})$} & {O(1)} & {$1+\alpha_{dirty-evict}\cdot 2 \cdot SPP$} \\\hline
        {LSM-tree} & {$r\cdot h \cdot FPR \cdot \alpha_{hit}$} & {$O(r\cdot h \cdot log_2(n_{run-entry}))$} & {$h$}\\\hline
        \end{tabular}
    \label{tab:plr-opt}
    \vspace{-10pt}
\end{table*}
}
\end{comment}